\section{Introduction}

Gravity imposes classicality---or so we have assumed. The Di\'{o}si-Penrose model gives this assumption mathematical form: spatial superpositions collapse at rate $\Gamma_0 = Gm^2/(\hbar d)$, with position as the preferred basis because gravity couples to mass distribution \cite{Penrose1996,Diosi1987}. Every gravitational collapse theory shares this premise \cite{Diosi1989,Ghirardi1990}. Every experiment has presupposed it \cite{Donadi2021,Hornberger2003,Delic2020}. None has tested it.

We test it here---theoretically---and predict it fails.

No experiment has varied the measurement basis while monitoring gravitational decoherence. We propose this as an unexplored degree of freedom and derive a basis-dependent rate:
%
\begin{equation}
\Gamma(\theta) = \Gamma_0 \cos^2\theta
\label{eq:main}
\end{equation}
%
where $\theta$ is the angle between measurement quadrature and position basis.

The reasoning is physical. Gravity couples to position. A position measurement ($\theta = 0°$) asks ``where?''---the question gravity tracks---and decoherence is maximal. A momentum measurement ($\theta = 90°$) asks ``how fast?''---orthogonal to gravity's domain---and decoherence vanishes. Gravity cannot decohere what it does not see. For a general quadrature $\hat{X}_\theta = \hat{x} \cos\theta + \hat{p} \sin\theta$, position content scales as $\cos\theta$; decoherence, an incoherent process, scales with probability: $\cos^2\theta$.

Quantitatively: for $m = 10^{-12}$~kg and $d = 1$~\textmu m, we predict $\tau = 0.14$~s (position), $0.28$~s ($45°$), and $\tau \to \infty$ (momentum). This is testable in existing levitated optomechanics setups by sweeping homodyne detection phase \cite{Aspelmeyer2014}---no new equipment required.

Confirmation would mean gravity does not select a preferred basis independently of observation. The classical world would emerge not from gravitational dynamics alone, but from the intersection of what gravity couples to and what we choose to measure. Refutation would validate the standard Di\'{o}si-Penrose assumption with the first direct test.

We develop the projection principle (\S\ref{sec:projection}), derive $\Gamma(\theta)$ (\S\ref{sec:derivation}), present predictions (\S\ref{sec:predictions}) and protocol (\S\ref{sec:protocol}), and discuss implications (\S\ref{sec:discussion}).
