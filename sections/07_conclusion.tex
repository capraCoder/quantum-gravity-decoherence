\section{Conclusion}
\label{sec:conclusion}

Gravitational decoherence rate, we predict, depends on measurement basis: $\Gamma(\theta) = \Gamma_0 \cos^2\theta$. This follows from the Di\'{o}si-Penrose master equation without modification. The question---what does an observer measuring a rotated quadrature see?---was not previously asked.

The prediction yields a clean signature. Sweeping homodyne phase from position ($\theta = 0°$) to momentum ($\theta = 90°$) should reveal coherence times that double at $45°$ and diverge at $90°$. The ratio $\tau(45°)/\tau(0°) = 2$ holds independent of mass, separation, or absolute rate---requiring only relative measurement.

All components exist: ground-state cooling, superposition protocols, phase-resolved detection. The experiment combines established techniques to test a prediction not previously considered. No new apparatus is required.

Confirmation implies gravity's preferred basis emerges at measurement. Refutation confirms standard Di\'{o}si-Penrose. Either outcome advances the quantum-gravitational frontier. The question is now empirical.
