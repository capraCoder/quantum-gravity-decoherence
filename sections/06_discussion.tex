\section{Discussion}
\label{sec:discussion}

The prediction $\Gamma(\theta) = \Gamma_0 \cos^2\theta$ has not appeared in prior literature. We address why, and consider the implications of confirmation or refutation.

\subsection{Why has this not been noticed?}

The Di\'{o}si-Penrose model was built to explain classicality---the emergence of definite positions from quantum superpositions. Its natural application is position measurement. No theory predicted basis-dependent decoherence, so no experiment tested for it. The gap is conceptual: the question was not asked.

\subsection{Relation to other collapse models}

Continuous spontaneous localization (CSL) and GRW \cite{Ghirardi1990,Bassi2003} also couple to position. Any collapse model with position-localization will exhibit $\cos^2\theta$ scaling when measured in rotated quadratures. The prediction is generic to position-coupled dynamics, not specific to gravitational origin.

\subsection{If confirmed}

A positive result would establish that gravitational decoherence depends on measurement basis---the rate $\Gamma$ is not intrinsic to the quantum state but emerges in the act of observation.

This echoes Wheeler's delayed-choice experiment \cite{Wheeler1978}, now realized in the laboratory \cite{Tang2012}: a photon's wave-or-particle character becomes definite only at detection. Basis-dependent gravitational decoherence would imply an analogous result: gravity's decohering action becomes definite only when---and in the basis where---we measure. The preferred basis emerges from the conjunction of dynamics and observation, not from gravity alone.

The full implications extend beyond collapse phenomenology. We develop them in subsequent work; here we note only that they exist.

\subsection{If refuted}

A null result---$\Gamma(\theta)$ constant across bases---confirms standard Di\'{o}si-Penrose. Gravitational decoherence proceeds at $\Gamma_0$ regardless of measurement basis. This rules out observer-dependence and constrains interpretations assigning measurement a constitutive role. Null results are not failures; they sharpen theoretical boundaries.

\subsection{Limitations}

Our derivation assumes the standard Di\'{o}si master equation with position coupling. Models with different operator structure would yield different angular dependence. The prediction tests position-coupled collapse specifically.

Realistic homodyne detection introduces inefficiency and noise, contributing to $\Gamma_{\text{env}}$. These do not alter the $\cos^2\theta$ signature; basis-dependent gravitational decoherence remains extractable from the fit.

\subsection{Future directions}

Extension to two-dimensional phase space yields $\Gamma(\theta, \varphi)$. Delayed-choice protocols---selecting measurement basis after superposition evolution---test whether basis-dependence is fixed at preparation or at detection. These directions are reserved for subsequent work.
