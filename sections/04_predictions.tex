\section{Predictions}
\label{sec:predictions}

The prediction $\Gamma(\theta) = \Gamma_0 \cos^2\theta$ becomes testable once we specify parameters. We calculate coherence times for systems within reach of current optomechanical technology.

\subsection{Di\'{o}si-Penrose decoherence rate}

For a sphere of mass $m$ and radius $R$ in superposition of separation $d \gg R$ \cite{Diosi1987,Diosi1989}:
%
\begin{equation}
\Gamma_0 = \frac{Gm^2}{\hbar R}
\label{eq:gamma0}
\end{equation}
%
Geometric corrections for finite $d/R$ appear in Ref.~\cite{Donadi2021}.

\subsection{Reference system}

We adopt parameters consistent with recent ground-state cooling experiments \cite{Delic2020}:

\begin{center}
\begin{tabular}{ll}
\hline
Parameter & Value \\
\hline
Material & Silica nanosphere \\
Mass & $10^{-12}$ kg \\
Diameter & $\sim$500 nm \\
Superposition separation & 1 $\mu$m \\
Mechanical frequency & $\omega_m/2\pi \sim 100$ kHz \\
\hline
\end{tabular}
\end{center}

Applying the Di\'{o}si formula: $\Gamma_0 \approx 7$ s$^{-1}$, giving $\tau_0 = 0.14$ s.

\subsection{Basis-dependent predictions}

From Eq.~\eqref{eq:main_result}:

\begin{center}
\begin{tabular}{lccc}
\hline
Measurement & $\theta$ & $\Gamma(\theta)$ [s$^{-1}$] & $\tau(\theta)$ [s] \\
\hline
Position & $0°$ & 7.0 & 0.14 \\
$45°$ quadrature & $45°$ & 3.5 & 0.28 \\
$60°$ quadrature & $60°$ & 1.75 & 0.57 \\
Momentum & $90°$ & 0 & $\infty$ \\
\hline
\end{tabular}
\end{center}

\subsection{The critical ratio}

The ratio $\tau(45°)/\tau(0°) = 2$ holds for any mass, separation, or $\Gamma_0$. It depends only on $\cos^2(45°) = 1/2$. This ratio is the cleanest experimental signature---independent of absolute calibration.

\subsection{Required precision}

Distinguishing $\tau(0°)$ from $\tau(45°)$ at $3\sigma$ requires coherence time uncertainty $< 20\%$. Current experiments achieve $\sim$10\% precision in decoherence measurements \cite{Vinante2019}. The challenge is not measurement precision but isolation: suppressing environmental decoherence until gravitational effects dominate.

\subsection{Current experimental status}

Ground-state cooling of levitated nanoparticles has been achieved \cite{Delic2020}. Sustained macroscopic superpositions at the 0.1 s timescale remain beyond current reach; environmental sources (blackbody radiation, residual gas, photon recoil) presently limit coherence to milliseconds \cite{Gonzalez2021}. Progress in cryogenic levitation and coherent scattering protocols suggests gravitationally-limited regimes may become accessible within this decade \cite{Aspelmeyer2014,Bassi2013}.

\subsection{Confirmation and refutation}

\begin{itemize}
\item \textit{Confirmation}: $\tau(\theta)$ varies with basis. A measured ratio $\tau(45°)/\tau(0°) = 2.0 \pm 0.4$ at $3\sigma$ would strongly support basis-dependence.
\item \textit{Refutation}: $\tau(\theta)$ constant across bases. This confirms standard Di\'{o}si-Penrose and rules out our prediction.
\end{itemize}

Both outcomes constrain fundamental physics. Neither is excluded by current theory.
