\section{The Projection Principle}
\label{sec:projection}

Measurement defines what can be observed. This is not philosophy---it is geometry.

Consider a pendulum on a rotating platform, viewed by cameras at different angles \cite{YourAJPPaper}. A camera aligned with the rotation axis cannot see axial displacement; a camera perpendicular to it sees full displacement. The pendulum's motion does not change. What the camera records does---because the measurement axis determines which component of motion appears in the data.

We call this the projection principle. It applies wherever observation geometry filters physical information.

In the Di\'{o}si-Penrose framework, gravity couples to mass distribution: position. The collapse mechanism monitors where the particle is---not how fast it moves, not its energy, not its spin. A measurement reveals gravitational decoherence only if it interrogates position. This constraint has been invisible because every experiment has measured position.

A homodyne detector measures the generalized quadrature $\hat{X}_\theta = \hat{x} \cos\theta + \hat{p} \sin\theta$. At $\theta = 0°$, it asks ``where?''---maximum overlap with what gravity monitors. At $\theta = 90°$, it asks ``how fast?''---zero overlap. A momentum measurement cannot see position-based decoherence, exactly as an axially-aligned camera cannot see axial motion.

No existing collapse model accounts for this. Standard treatments assume decoherence rate $\Gamma_0$ is fixed by mass and separation alone, independent of measurement basis \cite{Penrose1996,Diosi1987,Diosi1989,Ghirardi1990}. We propose instead that $\Gamma_0$ is the rate revealed when measuring position. Other measurements reveal a projected rate:
%
\begin{equation}
\Gamma(\theta) = \Gamma_0 \cos^2\theta
\label{eq:projection}
\end{equation}

The form follows from quantum measurement theory. The quadrature $\hat{X}_\theta$ has position-amplitude $\cos\theta$. Decoherence contributes to measurement statistics incoherently, scaling with probability---the square of amplitude. The factor $\cos^2\theta$ is not a model choice; it is a consequence of projection.
